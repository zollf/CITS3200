\documentclass[a4paper, draft]{article}

\author{Joel Milligan}
\title{Project Acceptance Test Manual}
\date{}

\begin{document}
\maketitle
\tableofcontents

\newpage
\section{Objectives}
In this section, describe the purpose of this document.
Forecast the testing strategy, the unit tests, integation test and system tests listed in this document.

\section{Document References}
In this section, reference the major documents produced during project development.

Explains the relationships among the requirements documents, design documents, implementation documents and the test procedures.

\section{Test Summary}
In this section, describe the functions of the system tested in this document.

Refer to Requirements Analysis Document and Problem Statement.

\section{Testing Strategy}
In this section, define the subsystem or subsystems to be tested, the system integration strategy and how, where, when, and by whom the tests will be conducted.

You may want to include drawings depicting relationships among the major classes of the subsystem or the subsystem decomposition, if you feel this is appropriate.

\section{Test Materials}
Describes materials required for executing the tests described in this document.

\newpage
\section{Test Example}
Introduction and overview for this test

\subsection{Test Specification}
The Test Specification lists the requirements whose satisfaction will be demonstrated by the test. It lists the methods tested, and describes the conditions of the test.

\subsection{Test Description}
The Test Description is used as a guide in performing the test. It lists the input data and input commands for each test, as well as expected out put and system messages.
If you find that you are unable to describe expected output numerically, use a natural language description.

\subsubsection{Test Description Contents}
\begin{itemize}
    \item Location of test (hyperlink to test)
    \item Means of Control: Describes how data are entered (manually or automatically with a test driver)
    \item Data
          \begin{itemize}
              \item Input Data
              \item Input Commands
              \item Output Data
              \item System Messages
          \end{itemize}
    \item Procedures: The test procedure is often specificed in form of a test script.
\end{itemize}

\subsection{Test Analysis Report}
The Test Analysis Report lists the functions and performance characteristics that were to be demonstrated, and describes the actual test results. The description of the results must include the following:
\begin{itemize}
    \item Function
    \item Performance
    \item Data measures, including whether target requirements have been met
\end{itemize}
If an error or deficiency has been discovered, the report discusses its impact.

\end{document}