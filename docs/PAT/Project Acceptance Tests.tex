\documentclass[a4paper, draft]{article}

\author{Joel Milligan}
\title{Project Acceptance Test Manual}
\date{}

\begin{document}
\maketitle
\tableofcontents

\newpage
\section{Objectives}
In this section, describe the purpose of this document.
Forecast the testing strategy, the unit tests, integation test and system tests listed in this document.

\section{Document References}
In this section, reference the major documents produced during project development.

Explains the relationships among the requirements documents, design documents, implementation documents and the test procedures.

\section{Test Summary}
In this section, describe the functions of the system tested in this document.

Refer to Requirements Analysis Document and Problem Statement.

\section{Testing Strategy}
In this section, define the subsystem or subsystems to be tested, the system integration strategy and how, where, when, and by whom the tests will be conducted.

You may want to include drawings depicting relationships among the major classes of the subsystem or the subsystem decomposition, if you feel this is appropriate.

\newpage
\section{Front End Unit Tests}
All front end JavaScript unit tests, using the \texttt{jest} framework, should pass.

\subsection{Test Procedure}
\begin{enumerate}
    \item Open your CLI to the root of the repository.
    \item Run the command \texttt{yarn test}.
\end{enumerate}

\subsection{Expected Results}
\begin{itemize}
    \item All unit tests pass.
    \item At least 50\% test coverage.
\end{itemize}
The reason for the relatively low coverage is that most tests for the frontend will be integration tests.

\section{Back End Unit Tests}
All back end Python unit tests, using \texttt{pytest}, should pass.

\subsection{Test Procedure}
\begin{enumerate}
    \item Open your CLI to the root of the repository.
    \item Run the command \texttt{yarn test}.
\end{enumerate}

\subsection{Expected Results}
\begin{itemize}
    \item All unit tests pass.
    \item At least 75\% test coverage.
\end{itemize}

\newpage
\section{Publically Accessible Website}
The booking part of the website should be accessible to anyone with a modern web browser.

\subsection{Requirements Tested}
A user who wants to make a booking should be able to able access the website so they can make a booking.

\subsection{Test Procedure}
\begin{enumerate}
    \item Navigate to the main website URL in either Google Chrome or Mozilla Firefox.
\end{enumerate}

\subsection{Expected Result}
The initial page for booking will be shown.
\end{document}